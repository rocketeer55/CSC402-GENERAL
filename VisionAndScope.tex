\documentclass[12pt,oneside,letterpaper]{article}
\begin{document}


\title{\bfseries Data Classifier: \\Vision and Scope}

\author {
\large{Team Mark}\\
\emph{Computer Science Department}\\
\emph{California Polytechnic State University}\\
\emph{San Luis Obispo, CA USA}\\
}

\date{October 8, 2018}
\maketitle \thispagestyle{empty}

\pagebreak
\tableofcontents

\addcontentsline{toc}{section}{Credits}

\section*{Credits}
\begin{tabular}{|l|l|p{2in}|l|}
\hline
\textbf{Name}&\textbf{Date}&\textbf{Role}&\textbf{Version}\\
\hline
Spencer Schurk&October 6, 2018&Lead Author of Business Requirements&1.0\\
\hline
&&&\\
\hline
&&&\\
\hline
&&&\\
\hline
\end{tabular}

\addcontentsline{toc}{section}{Revision History}

\section*{Revision History}
\begin{tabular}{|l|l|p{2in}|l|}
\hline
\textbf{Name}&\textbf{Date}&\textbf{Reason for Changes}&\textbf{Version}\\
\hline
Spencer Schurk&October 6, 2018&Initial version of Section 1&1.0\\
\hline
&&&\\
\hline
&&&\\
\hline
&&&\\
\hline
\end{tabular}


\newpage

\section{Business Requirements}
\subsection{Background}
Ever-growing data sets are causing an unnecessary burden on data scientists and data analysts. As data sets are expanding, and the number of sources where these sets come from is increasing, it is becoming harder for professionals working on this data to find the information they need. Much of this work currently is done manually, and makes it very difficult for someone new coming into a data position at an existing company with a large data sets to start working productively.
\subsection{Business Opportunity}
Developing a new data classifier tool will remove many of the manual hardships involved with analyzing large, and often unorganized data sets. Instead of spending hours searching for the proper data classification a data analyst might be looking for, the new data classifier tool will allow for automated classification. This tool would allow recently hired data analysts who don't have experience with a company's complex data sets to do analysis quicker. Developing this data classifier as a web-based interface allows data scientists and data analysts to spend less time looking for the data, and more time providing useful insights and metrics to the company.
\subsection{Business Objectives and Success Criteria}
\begin{tabular}{|p{1in}|p{4.5in}|}
\hline
\textbf{BO-1}&Develop a machine-learning powered data classifier.\\
\hline
\textbf{BO-2}&Develop a web-based GUI that interfaces with the data classifier and allows edits to classification.\\
\hline
\textbf{BO-3}&Data classifications can be exported into a Data Catalog.\\
\hline
\end{tabular}
\newline
\vspace{1cm}
\newline
\begin{tabular}{|p{1in}|p{4.5in}|}
\hline
\textbf{SC-1}&Increase productivity of data scientists and data analysts.\\
\hline
\textbf{SC-2}&Classifier scales over various data sets and successfully categorizes data.\\
\hline
\textbf{SC-3}&Open-source classifier sees adoption and adaptation by outside companies.\\
\hline
\end{tabular}
\subsection{Customer or Market Needs}
\begin{tabular}{|p{1in}p{4.5in}|}
\hline
\textbf{CN-1}&Interface should be viewable from a modern web-browser.\\
\hline
\textbf{CN-2}&Machine-Learning techniques should be used to classify incoming data.\\
\hline
\textbf{CN-3}&Users should be able to edit classifications before they're stored and sent to Data Catalog.\\
\hline
\end{tabular}
\subsection{Business Risks}
No known business risks at present.

\section{User Description}
\subsection{User/Market Demographics}
\subsection{User Personas}
\subsection{User Environment}
\subsection{Key User Needs}

\section{Vision of the Solution}
\subsection{Vision Statement}
\subsection{Solution Overview}
\subsection{Major Features}
\subsection{Assumptions and Dependencies}

\section{Scope and Limitations}
\subsection{Scope of Initial and Subsequent Releases}
Initial Release targets the end of 405.  One or two subsequent releases will occur in 406.
\subsection{Limitations and Exclusions}

\section{Business Context}
\subsection{Stakeholder Profiles}
\subsection{Project Priorities}
\subsubsection{Release 1}
\subsection{Operating Environment}

\section{Competitive Analysis}
\subsection{Overview}
What is the competition like?  Are there competitors?  How mature are they?  
\subsection{Competitor 1}
Describe a leading competitor.  Perhaps give a list of features they provide.  How do they market and price their product?
\subsection{Competitor n}
Repeat for as many leading competitors as is appropriate.

\end{document}
