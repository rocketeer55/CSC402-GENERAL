\documentclass[12pt,oneside,letterpaper]{article}
\begin{document}


\title{\bfseries Data Classifier: \\Vision and Scope}

\author {
\large{Team Mark}\\
\emph{Computer Science Department}\\
\emph{California Polytechnic State University}\\
\emph{San Luis Obispo, CA USA}\\
}

\date{October 8, 2018}
\maketitle \thispagestyle{empty}

\pagebreak
\tableofcontents

\addcontentsline{toc}{section}{Credits}

\section*{Credits}
\begin{tabular}{|l|l|p{2in}|l|}
\hline
\textbf{Name}&\textbf{Date}&\textbf{Role}&\textbf{Version}\\
\hline
Spencer Schurk&October 6, 2018&Lead Author of Business Requirements&1.0\\
\hline
Geraldo Macias&October 6, 2018&Lead Author of Scope and Limitations&1.1\\
\hline
&&&\\
\hline
&&&\\
\hline
&&&\\
\hline
\end{tabular}

\addcontentsline{toc}{section}{Revision History}

\section*{Revision History}
\begin{tabular}{|l|l|p{2in}|l|}
\hline
\textbf{Name}&\textbf{Date}&\textbf{Reason for Changes}&\textbf{Version}\\
\hline
Spencer Schurk&October 6, 2018&Initial version of Section 1&1.0\\
\hline
Geraldo Macias&October 6, 2018&Initial version of Section 4&1.1\\
\hline
&&&\\
\hline
&&&\\
\hline
&&&\\
\hline
\end{tabular}


\newpage

\section{Business Requirements}
\subsection{Background}
Ever-growing data sets are causing an unnecessary burden on data scientists and data analysts. As data sets are expanding, and the number of sources where these sets come from is increasing, it is becoming harder for professionals working on this data to find the information they need. Much of this work currently is done manually, and makes it very difficult for someone new coming into a data position at an existing company with a large data sets to start working productively.
\subsection{Business Opportunity}
Developing a new data classifier tool will remove many of the manual hardships involved with analyzing large, and often unorganized data sets. Instead of spending hours searching for the proper data classification a data analyst might be looking for, the new data classifier tool will allow for automated classification. This tool would allow recently hired data analysts who don't have experience with a company's complex data sets to do analysis quicker. Developing this data classifier as a web-based interface allows data scientists and data analysts to spend less time looking for the data, and more time providing useful insights and metrics to the company.
\subsection{Business Objectives and Success Criteria}
\begin{tabular}{|p{1in}|p{4.5in}|}
\hline
\textbf{BO-1}&Develop a machine-learning powered data classifier.\\
\hline
\textbf{BO-2}&Develop a web-based GUI that interfaces with the data classifier and allows edits to classification.\\
\hline
\textbf{BO-3}&Data classifications can be exported into a Data Catalog.\\
\hline
\end{tabular}
\newline
\vspace{1cm}
\newline
\begin{tabular}{|p{1in}|p{4.5in}|}
\hline
\textbf{SC-1}&Increase productivity of data scientists and data analysts.\\
\hline
\textbf{SC-2}&Classifier scales over various data sets and successfully categorizes data.\\
\hline
\textbf{SC-3}&Open-source classifier sees adoption and adaptation by outside companies.\\
\hline
\end{tabular}
\subsection{Customer or Market Needs}
\begin{tabular}{|p{1in}p{4.5in}|}
\hline
\textbf{CN-1}&Interface should be viewable from a modern web-browser.\\
\hline
\textbf{CN-2}&Machine-Learning techniques should be used to classify incoming data.\\
\hline
\textbf{CN-3}&Users should be able to edit classifications before they're stored and sent to Data Catalog.\\
\hline
\end{tabular}
\subsection{Business Risks}
No known business risks at present.

\section{User Description}
\subsection{User/Market Demographics}
\subsection{User Personas}
\subsection{User Environment}
\subsection{Key User Needs}

\section{Vision of the Solution}
\subsection{Vision Statement}
\subsection{Solution Overview}
\subsection{Major Features}
\subsection{Assumptions and Dependencies}



\newpage
\section{Scope and Limitations}
\subsection{Scope of Initial and Subsequent Releases}
405 initial release targets.
\begin{enumerate}
    \item Develop machine learning Python program which can detect different types of field types and classify different datasets.
    \item Create basic front end which can execute the python program on local datasets.
\end{enumerate}  
406 release 1 targets.
\begin{enumerate}
    \item The system will prompt the user to verify the contents and label of dataset columns. This edit will be applied and improve the machine learning on future datasets.
    \item The system will catalog the each dataset according to a type defined by the machine learning algorithm.
    \item A data scientist user may search for datasets which include specific fields, or search by types of datasets.
\end{enumerate}
406 release 2 targets.
\begin{enumerate}
    \item The system will allow account creations with different permission settings.
    \item The system will redact sensitive information according to account permission settings. A system administrator will have the highest privilege to information but no modification privileges. A data scientist will have some data redacted but modification privileges. An employee will have minimal data access and no modification privileges.
\end{enumerate}
\subsection{Limitations and Exclusions}
\begin{enumerate}
    \item All datasets will be of a .csv file type.
    \item Datasets must be stored within customers maintained database.
\end{enumerate}




\section{Business Context}
\subsection{Stakeholder Profiles}
\subsection{Project Priorities}
\subsubsection{Release 1}
\subsection{Operating Environment}

\section{Competitive Analysis}
\subsection{Overview}
What is the competition like?  Are there competitors?  How mature are they?  
\subsection{Competitor 1}
Describe a leading competitor.  Perhaps give a list of features they provide.  How do they market and price their product?
\subsection{Competitor n}
Repeat for as many leading competitors as is appropriate.

\end{document}
